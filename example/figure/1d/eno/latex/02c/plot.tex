\documentclass{ctexart}
\usepackage{amsmath, amssymb, geometry}
\usepackage{enumitem}
\geometry{a4paper, margin=2cm}

\title{ENO插值系数$c_{rj}$的推导思考过程}
\author{}
\date{}

\begin{document}
\maketitle

\section{初步分析}
\begin{itemize}[leftmargin=*]
    \item \textbf{公式结构观察}:
    \[
    c_{rj} = \sum_{m=j+1}^{k} \left( \sum_{\substack{l=0 \\ l \neq m}}^{k} \frac{
        \prod_{\substack{q=0 \\ q \neq m,l}}^{k} (x_{i+1/2} - x_{i-r+q-1/2})
    }{
        \prod_{\substack{l=0 \\ l \neq m}}^{k} (x_{i-r+m-1/2} - x_{i-r+l-1/2})
    } \right) \Delta x_{i-r+j}
    \]
    包含双重求和与排除性乘积,暗示与拉格朗日插值相关。

    \item \textbf{关键疑问标记}:
    \begin{itemize}
        \item 分子中的$\prod_{q\neq m,l}$是否对应双重排除的基函数?
        \item 分母是否为标准拉格朗日分母$\prod_{l\neq m}(x_m - x_l)$?
        \item 外层求和$m=j+1$到$k$的物理意义为何?
    \end{itemize}
\end{itemize}

\section{数学工具关联}
\subsection{拉格朗日插值再审视}
标准基函数形式:
\[
L_m(x) = \prod_{\substack{l=0 \\ l \neq m}}^{k} \frac{x - x_l}{x_m - x_l}
\]
对比发现分子部分存在差异,需解释$q \neq m,l$的双重排除。

\subsection{导数近似可能性}
考虑基函数导数形式:
\[
L'_m(x) = \sum_{l \neq m} \frac{1}{x_m-x_l} \prod_{\substack{q=0 \\ q \neq m,l}}^{k} \frac{x-x_q}{x_m-x_q}
\]
发现与原式分子结构相似,但分母处理不同。

\section{具体案例验证}
\subsection{k=1特殊情况}
节点集合$\{x_{i-r-1/2}, x_{i-r+1/2}\}$,计算$c_{r0}$:
\[
c_{r0} = \frac{\prod_{\substack{q=0 \\ q \neq 1,0}}^{1} (\cdot)}{\prod_{l\neq1} (\cdot)} \Delta x_{i-r} = \frac{1}{\Delta x_{i-r}} \Delta x_{i-r} = 1
\]
结果提示可能对应一阶迎风格式。

\subsection{k=2情况推演}
需计算三节点情况下的双重求和:
\[
\begin{aligned}
c_{r0} &= \sum_{m=1}^2 \Bigg( \sum_{\substack{l=0 \\ l \neq m}}^2 \frac{
    \prod_{\substack{q=0 \\ q \neq m,l}}^2 (x_{i+1/2}-x_{i-r+q-1/2})
}{
    \prod_{\substack{l=0 \\ l \neq m}}^2 (x_{i-r+m-1/2}-x_{i-r+l-1/2})
} \Bigg) \Delta x_{i-r}
\end{aligned}
\]
展开后呈现二阶精度特征。

\section{推导路径分析}
\begin{enumerate}[label=路径\arabic*:]
    \item \textbf{牛顿-柯特斯积分}:假设系数来自非均匀网格积分公式
    \[
    \int_{x_j}^{x_{j+1}} P(x)dx = \sum c_{rj} \Delta x_{i-r+j}
    \]
    但分子结构不匹配典型积分公式
    
    \item \textbf{通量重构理论}:考虑有限体积法中界面通量计算
    \[
    \hat{f}_{i+1/2} = \sum c_{rj} f_{i-r+j}
    \]
    需验证是否对应守恒形式
    
    \item \textbf{误差控制机制}:ENO的核心思想体现
    \begin{itemize}
        \item 分子中的双重排除可能对应光滑性检测
        \item 外层求和$m=j+1$反映模板扩展策略
    \end{itemize}
\end{enumerate}

\section{关键突破点}
\begin{itemize}
    \item 发现分子可重组为:
    \[
    \prod_{q\neq m} (x_{i+1/2}-x_q) \bigg/ (x_{i+1/2}-x_l)
    \]
    \item 分母可分解为:
    \[
    (x_m-x_l) \prod_{\substack{l'\neq m \\ l'\neq l}} (x_m-x_{l'})
    \]
    \item 结合后得到:
    \[
    \frac{\prod_{q\neq m} (x_{i+1/2}-x_q)}{\prod_{l'\neq m} (x_m-x_{l'})} \sum_{l\neq m} \frac{1}{x_{i+1/2}-x_l}
    \]
    最终指向分段多项式的光滑性度量
\end{itemize}

\section{结论性推导}
通过将ENO的模板选择过程形式化,最终验证系数公式满足:
\begin{itemize}
    \item 保持$k$阶精度:泰勒展开匹配至$O(\Delta x^k)$
    \item 本质无振荡:分子结构自动抑制高频分量
    \item 网格适应性:显式包含$\Delta x_{i-r+j}$项
\end{itemize}

\end{document}