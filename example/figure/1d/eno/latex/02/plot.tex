\documentclass{ctexart}
\usepackage{amsmath}
\begin{document}

\section{三阶精度边界值近似推导}

\subsection{问题描述}
已知单元格中心为 \(x_{i-1}, x_i, x_{i+1}\),对应的平均值为 \(\bar{u}_{i-1}, \bar{u}_i, \bar{u}_{i+1}\)。目标是用线性组合:
\[
u_{i+1/2} \approx a \bar{u}_{i-1} + b \bar{u}_i + c \bar{u}_{i+1}
\]
逼近右边界 \(x_{i+1/2}\) 处的值,并证明此近似具有三阶精度。

\subsection{推导过程}

\subsubsection{泰勒展开单元格平均值}
假设解 \(u(x)\) 充分光滑,以 \(x_i\) 为中心进行泰勒展开,单元格平均值的定义式为:
\[
\bar{u}_j = \frac{1}{\Delta x} \int_{x_j - \Delta x/2}^{x_j + \Delta x/2} u(x) dx
\]
对每个单元格进行展开(保留到三次项):

\begin{align}
\bar{u}_{i-1} &= \frac{1}{\Delta x} \int_{x_{i-1} - \Delta x/2}^{x_{i-1} + \Delta x/2} \left[ u_i - u'_i \Delta x + \frac{u''_i}{2} (\Delta x)^2 - \frac{u'''_i}{6} (\Delta x)^3 + \dots \right] dx \notag \\
&= u_i - u'_i \Delta x + \frac{13}{24} u''_i (\Delta x)^2 - \frac{5}{24} u'''_i (\Delta x)^3 + \mathcal{O}(\Delta x^4), \\
\bar{u}_i &= \frac{1}{\Delta x} \int_{x_i - \Delta x/2}^{x_i + \Delta x/2} \left[ u_i + u'_i (x - x_i) + \frac{u''_i}{2} (x - x_i)^2 + \dots \right] dx \notag \\
&= u_i + \frac{1}{24} u''_i (\Delta x)^2 + \mathcal{O}(\Delta x^4), \\
\bar{u}_{i+1} &= \frac{1}{\Delta x} \int_{x_{i+1} - \Delta x/2}^{x_{i+1} + \Delta x/2} \left[ u_i + u'_i \Delta x + \frac{u''_i}{2} (\Delta x)^2 + \frac{u'''_i}{6} (\Delta x)^3 + \dots \right] dx \notag \\
&= u_i + u'_i \Delta x + \frac{13}{24} u''_i (\Delta x)^2 + \frac{5}{24} u'''_i (\Delta x)^3 + \mathcal{O}(\Delta x^4).
\end{align}

\subsubsection{边界值的泰勒展开}
将 \(u_{i+1/2}\) 在 \(x_i\) 处展开:
\begin{equation}
u_{i+1/2} = u_i + \frac{1}{2} u'_i \Delta x + \frac{1}{8} u''_i (\Delta x)^2 + \frac{1}{48} u'''_i (\Delta x)^3 + \mathcal{O}(\Delta x^4).
\end{equation}

\subsubsection{匹配系数}
将线性组合 \(a \bar{u}_{i-1} + b \bar{u}_i + c \bar{u}_{i+1}\) 代入泰勒展开式,要求其与 \(u_{i+1/2}\) 的展开一致。比较各阶项的系数:

\begin{align}
\text{常数项:} & \quad a + b + c = 1, \\
\text{一阶项:} & \quad (-a + c) \Delta x = \frac{1}{2} \Delta x \ \Rightarrow \ -a + c = \frac{1}{2}, \\
\text{二阶项:} & \quad \frac{13a + b + 13c}{24} (\Delta x)^2 = \frac{1}{8} (\Delta x)^2 \ \Rightarrow \ 13a + b + 13c = 3, \\
\text{三阶项:} & \quad \frac{-5a + 5c}{24} (\Delta x)^3 = \frac{1}{48} (\Delta x)^3 \ \Rightarrow \ -5a + 5c = \frac{1}{2}.
\end{align}

联立方程组:
\[
\begin{cases}
a + b + c = 1, \\
-a + c = \frac{1}{2}, \\
13a + b + 13c = 3, \\
-5a + 5c = \frac{1}{2}.
\end{cases}
\]
解得唯一解:
\[
a = -\frac{1}{6}, \quad b = \frac{5}{6}, \quad c = \frac{1}{3}.
\]

\subsubsection{误差分析}
将系数代入线性组合,计算误差:
\[
\text{误差} = \left(-\frac{1}{6} \bar{u}_{i-1} + \frac{5}{6} \bar{u}_i + \frac{1}{3} \bar{u}_{i+1}\right) - u_{i+1/2}.
\]
展开后误差项为:
\[
\left(\frac{-5a + 5c}{24} - \frac{1}{48}\right) u'''_i (\Delta x)^3 = \left(\frac{5}{36} - \frac{1}{48}\right) u'''_i (\Delta x)^3 = \mathcal{O}(\Delta x^3).
\]
因此,近似具有三阶精度。

\subsection{结论}
最终重构公式为:
\[
u_{i+1/2} = -\frac{1}{6} \bar{u}_{i-1} + \frac{5}{6} \bar{u}_i + \frac{1}{3} \bar{u}_{i+1} + \mathcal{O}(\Delta x^3).
\]
\end{document}