\documentclass{article}
\usepackage{amsmath}
\usepackage[UTF8]{ctex} % For Chinese support

\begin{document}

\section{8.1}

\begin{equation}
w_t + (f(w))_x = 0, \quad x \in (a, b), \quad t > 0
\label{eq:8.1}
\end{equation}

\begin{equation}
w(x, 0) = w^0(x), \quad x \in [a, b]
\label{eq:8.2}
\end{equation}

\begin{equation}
a = x_{\frac{1}{2}} < x_{\frac{3}{2}} < \cdots < x_{N-\frac{1}{2}} = b
\label{eq:8.3}
\end{equation}

其中计算点为 \( x_i = \frac{x_{i-1/2} + x_{i+1/2}}{2} \),步长为 \( \Delta x_i \)。  
\( \Delta x = \max \Delta x_i \),表示最大网格步长。

加权ENO方法是一种高阶、非振荡的数值方法,适用于求解偏微分方程。

\end{document}